\documentclass[10pt]{article}
\usepackage[margin=1in]{geometry}
\usepackage{multicol}
%\usepackage{setspace}
%\usepackage{amstext}
%\usepackage{amsmath}
%\usepackage{enumerate}
%\usepackage{graphicx}
%\usepackage{wrapfig}

\title{
	\textbf{
		Creating a Turn-Based Conflict Resolution Simulator
	}
}
\author{Noah Zimmt and Dakota Szabo}
\date{May 8, 2012}

\begin{document}
	\maketitle
	\begin{abstract}
		Lorem ipsum dolor sit amet, consectetur adipiscing elit. Duis vitae velit in sem aliquet tempus. Nam ullamcorper condimentum magna. Nam eu ligula sed velit mattis luctus quis sit amet felis. Proin fringilla venenatis lorem non scelerisque. Ut eu elit diam, eget fringilla nisl. Sed placerat dolor in turpis imperdiet sit amet dictum odio vehicula. Donec ultricies porttitor velit, sed tempus lectus commodo eu.
	\end{abstract}

	\begin{multicols}{2}
		\section*{Introduction}
		

		\section*{Related Work}
		Proin scelerisque urna et velit rutrum non feugiat nibh luctus. Aliquam posuere viverra lectus ut varius. Nam at erat tellus. Donec sed consectetur felis. Praesent at justo sit amet est vehicula pharetra viverra et magna. Nulla dignissim consectetur facilisis. Vestibulum a dui ligula. Nullam ornare sollicitudin molestie. Integer fringilla lacus ut metus lobortis mollis. Fusce magna velit, pulvinar a molestie vel, cursus et lorem. Phasellus semper pellentesque turpis eget faucibus. Proin at neque eu neque elementum sollicitudin vitae sit amet nunc. Etiam nec erat ut enim sagittis facilisis. Curabitur sollicitudin convallis arcu sit amet dignissim. Pellentesque habitant morbi tristique senectus et netus et malesuada fames ac turpis egestas.
		
		\section*{Rules and Regulations}
		Our simulation models the popular domination game \emph{Risk}. The map is laid out as a graph (a set of vertices and edges) in which each territory corresponds to a node in the graph and each border between territories corresponds to an edge between two nodes. While the graphs we used to model simulations were complete graphs, the simulator is capable of simulating conflicts given any user-provided graph and set of initial troop count.  

		Each territory must manage its army (a collection of troops), placing troops on edges to attack other territories or defend itself against other territories. Each territory belongs to a team; a territory will adopt its team's strategy for placing troops on edges, allowing the simulator to show how successful a strategy will be given graph shape and initial troop counts.

		The simulator is turn-based, with two phases comprising each turn: the decision stage and the resolution stage. During the decision stage, each territory chooses to place troops on borders; a territory must choose how many of its troops are to be placed on each of its borders and what action (attack or defend) the collection of troops will perform on that border. 

		
		
		\section*{Performance Analysis}
		Proin scelerisque urna et velit rutrum non feugiat nibh luctus. Aliquam posuere viverra lectus ut varius. Nam at erat tellus. Donec sed consectetur felis. Praesent at justo sit amet est vehicula pharetra viverra et magna. Nulla dignissim consectetur facilisis. Vestibulum a dui ligula. Nullam ornare sollicitudin molestie. Integer fringilla lacus ut metus lobortis mollis. Fusce magna velit, pulvinar a molestie vel, cursus et lorem. Phasellus semper pellentesque turpis eget faucibus. Proin at neque eu neque elementum sollicitudin vitae sit amet nunc. Etiam nec erat ut enim sagittis facilisis. Curabitur sollicitudin convallis arcu sit amet dignissim. Pellentesque habitant morbi tristique senectus et netus et malesuada fames ac turpis egestas.
		
		
		\section*{Conclusion}
		Nullam tortor mi, volutpat quis auctor non, aliquam id tellus. Morbi a massa libero, id tempus mi. Aliquam et sapien non est mollis interdum. Cras felis diam, luctus eget pellentesque ut, lacinia nec dui. Aenean eu orci neque, eu volutpat enim. Sed eleifend fringilla turpis, sed pharetra libero mollis non. Quisque ultrices, purus ut sagittis euismod, orci risus venenatis lacus, a auctor diam augue vitae dolor. Duis eleifend pulvinar enim ac mollis. Nulla auctor metus vel turpis consequat aliquam.
		
		\section*{Future Work}
		Quisque velit neque, ullamcorper non iaculis sed, ornare pellentesque nulla. In pharetra cursus imperdiet. Duis nulla justo, bibendum et consequat congue, condimentum et turpis. Sed tempus, odio ut posuere volutpat, quam lacus interdum odio, ac porttitor elit turpis vitae turpis. Mauris eu ultrices augue. Sed fermentum elementum tristique. Nam ipsum ipsum, rhoncus vitae cursus suscipit, volutpat eu felis.

	\end{multicols}
\end{document}